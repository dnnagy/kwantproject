\documentclass[12pt]{article}
%\documentclass[12pt,a4paper]{article}
%Size -- the best choice for university course
%slides seems landscape A5, 12pt letters.  Margins
%are made narrow, as much text on each page as possible.

%Could define size explicitly, but finally A5 was selected:

\usepackage[%paperwidth=600pt,paperheight=450pt,
a5paper,left=15pt,right=15pt,top=3pt,bottom=3pt,%
  %outer=25mm,
  %inner=35mm,
  %vmargin=2mm,
  includehead,includefoot,headheight=15.4pt,headsep=5pt,%
  footskip=22pt,landscape]{ geometry }


\usepackage{tocloft}
%\usepackage{tocloft}% http://ctan.org/pkg/tocloft
\setlength{\cftsecnumwidth}{1.9em}
\setlength{\cftsubsecnumwidth}{2.95em}
\setlength{\cftsubsubsecnumwidth}{3.7em}
% Set length of number width in ToC
% so that the number has enough space and
% does not run into the subsection title.

\usepackage[unicode,colorlinks=true,breaklinks]{hyperref}
%\usepackage[dvips]{hyperref}
%should display links, but it does not work with \H accent
%and formulas in section titles

\hypersetup{colorlinks,%
  linkcolor=blue,urlcolor=magenta,citecolor=green}
%Breaks long url`s in text, while keeping it one link:
\usepackage{breakurl}

\newcommand{\nc}{\newcommand} % a newcommand rövidítése
\newcommand{\rc}{\renewcommand} % a renewcommand rövidítése

\newcommand{\bb}[1]{\mathbf{#1}}
\newcommand{\dd}{\mathrm{d}}

%Creates header and footer w time, section titles, etc. 
%Here the stripes are made narrow. 
\usepackage{fancyhdr}
\usepackage[yyyymmdd,hhmmss]{datetime}

\pagestyle{fancy}
\rc{\headrulewidth}{0.3pt}
\rc{\footrulewidth}{0.3pt}
\lfoot{\today\, \currenttime}
\cfoot{}
\rfoot{\thepage}

\sloppy %To prefer larger spaces between 

%words rather than overflow a line. 
\frenchspacing %No wide space after period, this is the
%French and also the Hungarian rule. 

\usepackage{bookmark}
\usepackage[medium,compact,toctitles]{titlesec}
\usepackage[active]{srcltx}
\usepackage{braket}
%\usepackage{savetrees} 
% others: 

\usepackage{t1enc}
\usepackage{amsmath, amssymb, amsfonts, amsbsy, amsthm}
\usepackage{mathtools}
\usepackage{soul} % For \st command
\let\equation\gather
\let\endequation\endgather

%\usepackage[magyar]{babel}
\usepackage[utf8]{inputenc} 
%\usepackage[pdftex]{graphicx}

\usepackage{graphicx}
%\graphicspath{{./}{kepek/}{./files/}}

\usepackage{epstopdf}
\epstopdfsetup{update}
% only regenerate pdf files when eps file is newer



\usepackage[width=.75\textwidth]{caption}
\usepackage{lmodern}%Avoid rastering for letters.  
 
% Alább definiáljuk a magyar névelőhöz illeszkedő egyenlet hivatkozást.
% A magyar csomag része \az{..} az argumentumbeli számnévhez 
% illeszkedő névelő, \Az ugyanez % nagybetűvel, ezeket használjuk 
% alább.  
\nc{\Aeqref}[1]{\Az{\eqref{#1}}}
\nc{\aeqref}[1]{\az{\eqref{#1}}}

\usepackage{bbold}
\usepackage{ulem}
\usepackage[caption=false]{subfig} 
%\usepackage{epsfig}
%\usepackage[dvips]{color}
\usepackage{accents}
\usepackage{bbold}
\usepackage{textcomp}
\usepackage{nicefrac}%\nicefrac{}{} makes frac with slash for text
\usepackage[section]{placeins} %have \FloatBarrier,
%issued automatically at section titles 

\usepackage{wrapfig} %for wrapping text around floats 
\usepackage{indentfirst}
%for indenting the first para after section title, customary in 
%Hungarian but not so in English.  
\usepackage[integrals]{wasysym} %symbols for planets 
\usepackage{upgreek}
\usepackage{bm}\usepackage{bbm}
%I experimented with those:
%\usepackage[utopia]{mathdesign}
%\usepackage[OMLmathrm,OMLmathbf]{isomath}
%\usepackage{movie15}
%\usepackage[thinlines]{easytable} % DOES NOT WORK!
%comfortable table creation
%\begin{TAB}(r,1cm,2cm)[5pt]{|l|}{|c|c|}
%(rows,min,max)[tabcolsep]{columns}{rows}
%\end{TAB}


\rc*\familydefault{\sfdefault}
%entire doc in sanserif for better reading from projection

%redefine paragraph to enumerate them
\setcounter{secnumdepth}{6}
\rc\theparagraph{{\it\bf\alph{paragraph}}}
\rc\thesubparagraph{{\it\bf\roman{subparagraph}}}

\makeatletter
\rc\paragraph{\@startsection{paragraph}{4}{\z@}
{-3.25ex\@plus -1ex \@minus -.2ex}%
{5pt \@plus .2ex}%
{\normalfont\normalsize\bfseries\itshape}}
\rc\subparagraph{\@startsecton{paragraph}{5}{\z@}%
{-3.25ex\@plus -1ex \@minus -.2ex}%
{5pt \@plus .2ex}%
{\normalfont\normalsize\itshape}}

% \rc*\itemize{%
%   \ifnum \@itemdepth >\thr@@\@toodeep\else
%     \advance\@itemdepth\@ne
%     \edef\@itemitem{labelitem\romannumeral\the\@itemdepth}%
%     \expandafter
%     \list
%       \csname\@itemitem\endcsname
%       {\def\makelabel##1{\hss\llap{##1}}%
%        \setlength{\itemsep}{2pt}%
%        \setlength{\parsep}{2pt}%
%        \setlength{\parindent}{7pt}}%
%   \fi}

\makeatother

%\rc{\labelitemi}{$\to$}
%\rc{\labelitemii}{$\star$}



%Az "innen következik" jel formulában 
% \nc{\ra}{ \ \ \leadsto \ \ } %Ezen kicsi a nyíl.

% \nc{\el}{\mathcal{E}}

%számozott műfajok 
%\newtheorem{hf}{\text{\it Házi feladat}}[section]
%\nc{\HF}[1]{{\hf \tc{MB}{#1}}}
% \newtheoremstyle{mystyle}
% {headspec} \thmnumber{ #1} \thmname{#2} \thmnote{#3}
% \newtheorem{gf}{\text{\it{\cya{Gyakorló feladat}}}}[section]
% \nc{\GF}[1]{{\gf #1}}
% \nc{\GFL}[2]{{\begin{gf}\label{gf:#2} #1\end{gf}}}
% \nc{\GFN}[1]{{\vspace{-0.5\baselineskip}\gf #1}}
% \newtheorem{pl}{\text{\it{{Példa}}}}[subsection]

% \usepackage{xpatch}
% \makeatletter
% \xpatchcmd{\@thm}{\thm@headpunct{.}}{\thm@headpunct{}}{}{}
% \makeatother

% \nc{\PL}[1]{{\pl \gre{#1}}}
% \nc{\PLL}[2]{{\begin{pl}\label{pl:#2} \gre{#1}\end{pl}}}
% \nc{\dl}[2]{\hfill\yel{$2016.#1  \ \blacktriangleleft \ %
% \mid \ \blacktriangleright \ 2016.#2$}}%az előadások dátumai
% \nc{\dln}[2]{\hfill\yel{$2017.#1  \ \blacktriangleleft \ %
% \mid \ \blacktriangleright \ 2017.#2$}}%az előadások dátumai



%praktikus: vb soremelés, vbn állítható, nn egyenlet végére
\nc{\vb}{\vspace{\baselineskip}}
\nc{\vbn}[1]{\vspace{#1\baselineskip}}
\nc{\nn}{\nonumber}

%\numberwithin{equation}{subsection}
\numberwithin{equation}{section}
\usepackage{cleveref}

\setlength{\parskip}{0.3\baselineskip}
%\setlength{\itemsep}{-0.3\baselineskip} --- does not work
%\setlength{\itemsep}{2pt plus2pt minus2pt} --- does not work
\AtBeginDocument{\addtocontents{toc}{\protect\setlength{\parskip}{0pt}}}
%do not loosen toc by possibly large parskip
\linespread{0.93} %compactify

%compactify
\setcounter{topnumber}{2}
\setcounter{bottomnumber}{2}
\setcounter{totalnumber}{4}
\rc{\topfraction}{0.85}
\rc{\bottomfraction}{0.85}
\rc{\textfraction}{0.15}
\rc{\floatpagefraction}{0.7}
\rc{\theHequation}{\theHsection\arabic{equation}}
%needed for hyperref


% \usepackage{movie15} % to include movies
% \usepackage{hyperref} % prerequisite for movie15 -- these do not seem 
% to work includeonly{em0.1,em1.8} % selects from include list below, 
% but uses the other aux files and keeps eq. and sec. numberings mostly 
% correct; omitting includeonly results in including all items -- then 
% \include is equivalent to \clearpage\input{file}\clearpage.
% each included file gets its own .aux file by include
% \include{em0.1} etc.

% Below we use \input, very simple.
\begin{document}

\begin{center}

\thispagestyle{empty}

\rule{10 cm}{2pt}

\vspace{1.5cm}
{ \Large Numerical simulation of quantum transport phenomena \\ using the kwant package}

\vspace{1.0cm}
Nagy Dániel

\vspace{0.5cm}
Final Presentation \\
16 May 2019

\vspace{0.5cm}
%\today

\vspace{0.25cm}
\rule{10 cm}{2pt}

\end{center}
\newpage

%\setcounter{page}{2}
%\tableofcontents
\newpage
\phantomsection

\newpage
\section{Schedule for the semester}
\begin{table}[ht]
  \centering
  \caption{Status}
  \begin{tabular}{|c|c|}
  \hline
  Week & Scheduled Task \\ [0.5ex]  \hline % inserts table %heading
  feb. 18. - feb. 24. & Installing Kwant \& Running an example  \checkmark \\ \hline
  feb. 25. - mar. 3. & Reading the documentation \& Running more examples \checkmark \\ \hline
  mar. 4 - mar. 10 & Struggling with graphene minimal conductivity - no result \\ \hline 
  mar. 11. - mar. 17. & 2DEG basics \& Eigenstates and LDOS calculation \checkmark \\ \hline
  mar. 18. -  mar. 24. & 2DEG in magnetic field \checkmark \\ \hline
  mar. 25. -  mar. 31. & Mid term report \checkmark \\ \hline
  apr. 1. -  apr. 7. & Topological Anderson Insulator 1. \checkmark \\ \hline
  apr. 8. -  apr. 14. & Topological Anderson Insulator 2. \checkmark \\ \hline
  easter holiday & - \\ \hline
  apr. 22. - apr. 28. & Topological Anderson Insulator 3. \checkmark \\ \hline 
  apr. 29. - may 5. & Topological Anderson Insulator 4. \checkmark \\ \hline 
  Eötvös/Pázmány days & - \\ \hline
  may 13. - may 19. & Final report \checkmark \\ \hline
  \end{tabular}
\end{table}

\newpage

\section*{Introduction}
\begin{itemize}
    \item What is Kwant?
    \subitem Kwant is a free (open source), powerful, and easy to use Python package for numerical calculations on tight-binding models with a strong focus on quantum transport \cite{kwantpage,kwant-paper}
\end{itemize}

\newpage
\begin{itemize}
    \item What You do with Kwant?
        \subitem You can define a general tight binding system: square lattice, bipartite lattice, graphene, etc.
        \subitem You can attach leads to the scattering region, then calculate transport properties
        \subitem You can calculate densities (spin-density, charge-density, density of states, etc.)
        \subitem You can calculate eigenvalues, eigenfunctions
        \subitem You can calculate currents (spin-current, charge current, etc.)
        \subitem Many more: \cite{kwantdoc}.
\end{itemize}

\newpage
\section*{Defining a simple square lattice}
\begin{figure}[h!]
  \centering
  \begin{minipage}{0.49\textwidth}
    \centering
    \includegraphics[width=1.0\textwidth]{./media/example-1-code.png} % second figure itself
    %\caption{second figure}
\end{minipage}\hfill
  \begin{minipage}{0.49\textwidth}
      \centering
      \includegraphics[width=1.0\textwidth]{./media/example-1.png} % first figure itself
      %\caption{first figure}
  \end{minipage}
\end{figure}

\newpage
\section*{Defining a circular disc of square lattice}
\begin{figure}[h!]
  \centering
  \begin{minipage}{0.49\textwidth}
    \centering
    \includegraphics[width=1.0\textwidth]{./media/example-circle-code.png} % second figure itself
    %\caption{second figure}
\end{minipage}\hfill
  \begin{minipage}{0.49\textwidth}
      \centering
      \includegraphics[width=1.0\textwidth]{./media/example-circle.png} % first figure itself
      %\caption{first figure}
  \end{minipage}
\end{figure}


\newpage
\section*{Calculating the eigenfunctions}
\begin{figure}[h!]
  \centering
  \begin{minipage}{0.49\textwidth}
    \centering
    \includegraphics[width=1.0\textwidth]{./media/example-eigenfunc-code.png} % second figure itself
    %\caption{second figure}
\end{minipage}\hfill
  \begin{minipage}{0.49\textwidth}
      \centering
      \includegraphics[width=1.0\textwidth]{./media/example-eigenfunc.png} % first figure itself
      %\caption{first figure}
  \end{minipage}
\end{figure}

\newpage
\section*{Calculating the transmission for a quantum point contact}
\begin{figure}[h!]
    \begin{center}
        \includegraphics[height=0.7\textheight]{./media/2degconst_W8_L10.png}
        \caption{2DEG system: a point contact}
        \label{fig:2degconst}
    \end{center}
\end{figure}

\newpage
\section*{Calculating the transmission for a quantum point contact}
\begin{figure}[h!]
    \begin{center}
        \includegraphics[height=0.7\textheight]{./media/2deg_conductivity_W40_L20_VL6_0_VS6_0.png}
        \caption{Transmission coefficient for a quantum point contact plotted against the energy.}
        \label{fig:2deg_conductivity_W40_L20_VL6.0_VS6.0.png}
    \end{center}
\end{figure}

\newpage
\section*{Calculating the shot-noise for a quantum point contact}
\begin{figure}[h!]
    \begin{center}
        \includegraphics[height=0.7\textheight]{./media/2deg_shotnoise_W40_L20_VL6_0_VS6_0.png}
        \caption{Shot-noise for a quantum point contact calculated with kwant.}
        \label{fig:2deg_shotnoise_W40_L20_VL6.0_VS6.0.png}
    \end{center}
\end{figure}

\newpage 
\section*{Adding magnetic field}
If the system is placed in magnetic field, according to Peierls, we have to replace the hopping matrix elements:
\begin{equation*}
  t_{ij} \rightarrow t_{ij} \times \exp\left(i \frac{e}{\hbar} \int_{\mathbf{x}_j}^{\mathbf{x}_i} \mathbf{A}(\mathbf{x}) d\mathbf{s}\right)
\end{equation*}
For 2D square lattices this can be written as 
\begin{equation*}
  \exp\left(i\, 2 \pi \frac{\phi}{\phi_0} \frac{(y_i + y_j)(x_i-x_j)}{2a^2} \right)  
\end{equation*}
where $\phi = B a^2$ is the flux through a unit cell in the square lattice, and $\phi_0 = h/e$ the flux quantum. 

\newpage
\section*{Adding magnetic field}
\begin{figure}[h!]
  \begin{center}
  \includegraphics[height=0.55\textheight]{./media/example-peierls-code.png}
  \caption{Adding Peierls-hoppings to the system}
  \label{fig:peierls}
  \end{center}
\end{figure}

\newpage
\section*{Adding magnetic field}
\begin{figure}[h!]
    \begin{center}
        \includegraphics[height=0.7\textheight]{./media/qpc_L10_W5_VS0_2_cond_phi_e1_5.png}
        \caption{Transmission coefficient of a quantum point contact in function of the applied magnetic field}
        \label{fig:qpc_L10_W5_VS0_2_cond_phi_e1_50.png}
    \end{center}
\end{figure}

\newpage
\section*{Graphene minimal conductivity}
\begin{figure}[h!]
    \begin{center}
        \includegraphics[height=0.7\textheight]{./media/graphene_layer_W=20L=20.png}
        \caption{Graphene lattice defined in kwant}
        \label{fig:graphene_layer_W=20L=20.png}
    \end{center}
\end{figure}

\newpage
\section*{Graphene minimal conductivity}
\begin{figure}[h!]
    \begin{center}
        \includegraphics[height=0.7\textheight]{./media/graphene_conductivity_WL.png}
        \caption{Graphene conductivity in function of $W/L$ ratio.}
        \label{fig:graphene_conductivity_WL.png}
    \end{center}
\end{figure}

\newpage
\section*{Graphene minimal conductivity}
\begin{figure}[h!]
    \begin{center}
        \includegraphics[height=0.7\textheight]{./media/graphene_minimal_vg_l_0-3.png}
        \caption{Graphene conductivity in function of energy.}
        \label{fig:graphene_minimal_vg_l_0-3.png}
    \end{center}
\end{figure}

\newpage
\section*{Topological Anderson Insulator}
What are topological Andreson insulators?
Main article \cite{andersoninsulator}: \url{https://arxiv.org/abs/0811.3045},

\begin{minipage}[c]{0.45\textwidth}
  \begin{itemize}
    \item The physics of a topological insulator is unaffected by weak
    disorder, but is destroyed by large disorder
    \item BUT: Disorder can create a topological insulator for parameters where the system was metallic in 
    the absence of disorder
    \item These states are called topological Anderson insulators
    \item Disorder can be modeled as random on-site energy with a uniform distribution within $[-W/2, W/2]$
    \item the article discusses disordered strips of HgTe/CdTe quantum wells.
    \item Right: Results for a HgTe TAI, from the article \cite{andersoninsulator}. The plateau starting at
    $W=100, E_f=20\textnormal{meV}$ is a signature of the Topological Anderson Insulator.
  \end{itemize}
\end{minipage}
\hfill
\begin{minipage}[c]{0.5\textwidth}
    \includegraphics[width=\textwidth]{./media/tai_results_from_article.png}
\end{minipage}

\newpage
\section*{Topological Anderson Insulator}

\begin{itemize}
    \item The paper about Topological Anders Insulator (\url{https://arxiv.org/abs/0811.3045}) defines the 
    Hamiltonian for HgTe/CdTe heterostructure:
    \begin{equation*}
      \mathcal H(\bb k) = \left(
        \begin{array}{cc}
          h(\bb k) & 0 \\
          0 & h^{\dagger}(-\bb k) 
        \end{array}
        \right)
    \end{equation*}
    \begin{equation*}
      h(\bb k) = \epsilon(k) + \bb d (\bb k)\cdot\bb\sigma \textrm{, } \, \bb k = (k_x, k_y) 
    \end{equation*}
    \begin{equation*}
      \bb d(\bb k ) = (Ak_x, Ak_y, M-Bk^2) \textrm{; } \, \epsilon(k) = C-Dk^2
    \end{equation*}
  
    \item This can be rewritten as a sum of a few terms:
    \begin{equation*}
      =(C-Dk^2)\left(
    \begin{array}{cc}
    \sigma_0 & 0 \\
    0 & \sigma_0
    \end{array}
    \right)
    + (M-Bk^2)
    \left(
    \begin{array}{cc}
    \sigma_z & 0 \\
    0 & \sigma_z
    \end{array}
    \right)
    + Ak_x 
    \left(
    \begin{array}{cc}
    \sigma_x & 0 \\
    0 & -\sigma_x
    \end{array}
    \right)
    + Ak_y
    \left(
    \begin{array}{cc}
    \sigma_y & 0 \\
    0 & -\sigma_y
    \end{array}
    \right)
    + V_{\textrm{disorder}} 
    \left(
    \begin{array}{cc}
    \sigma_0 & 0 \\
    0 & \sigma_0
    \end{array}
    \right)
    \end{equation*}
  
    \item A, B, C, D, M are empirical parameters. I used the values presented in an article \cite{andersoninsulator}.
\end{itemize}


\newpage
\section*{Topological Anderson Insulator}
\begin{figure}[h!]
    \begin{center}
        \includegraphics[width=0.75\textwidth]{./media/hamilton_def.png}
    \end{center}
\end{figure}

\newpage
\section*{Topological Anderson Insulator}
\begin{figure}[h!]
    \begin{center}
        \includegraphics[height=0.75\textheight]{./media/bstructure_params_Lx=1500_Ly=500_M=-0_001.png}
    \end{center}
    \caption{Band structure for the "inverted" quantum well, with $M=-1\textnormal{meV}$}
\end{figure}

\newpage
\section*{Topological Anderson Insulator}
\begin{figure}[h!]
    \begin{center}
        \includegraphics[height=0.75\textheight]{./media/bstructure_params_Lx=1500_Ly=500_M=0_012.png}
    \end{center}
    \caption{Band structure for the "normal" quantum well, with $M=12\textnormal{meV}$}
\end{figure}


\newpage
\section*{Topological Anderson Insulator}
\begin{figure}[h!]
    \begin{center}
        \includegraphics[height=0.75\textheight]{./media/tai_results_params1_nsamples=4.png}
    \end{center}
    \caption{Conductivity plotted against disorder parameter for $M=12\textnormal{meV}$}
\end{figure}

\newpage
\section*{Topological Anderson Insulator}
\begin{figure}[h!]
  \centering
  \begin{minipage}{0.49\textwidth}
    \centering
    \includegraphics[width=1.0\textwidth]{./media/tai_results_from_article.png} % second figure itself
    %\caption{second figure}
\end{minipage}\hfill
  \begin{minipage}{0.49\textwidth}
      \centering
      \includegraphics[width=1.0\textwidth]{./media/tai_results_params1_nsamples=4.png} % first figure itself
      %\caption{first figure}
  \end{minipage}
\end{figure}

\newpage
\nocite{*}
\bibliographystyle{plain}
\bibliography{references}

\newpage
\topskip0pt
\vspace*{\fill}
\begin{center}
    \LARGE{Thank You for listening!}
\end{center}
\vspace*{\fill}


\end{document}
